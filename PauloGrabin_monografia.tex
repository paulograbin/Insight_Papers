\documentclass[english,brazilian]{UNISINOSmonografia}
\usepackage[utf8]{inputenc} % charset do texto (utf8, latin1, etc.)
\usepackage[T1]{fontenc} 	% encoding da fonte (afeta a sep. de sílabas)
\usepackage{graphicx} 		% comandos para gráficos e inclusão de figuras
\usepackage{bibentry} 		% para inserir refs. bib. no meio do texto
\usepackage{placeins}		% float barriers
\usepackage{enumitem}

%=======================================================================
% A vantagem do unisinos.bst é que ele permite o uso de um arquivo .bib
% seguindo as orientações tradicionais do BibTeX (veja essas orientações
% em http://ctan.tug.org/tex-archive/biblio/bibtex/contrib/doc/btxdoc.pdf).
%=======================================================================
\unisinosbst

%=======================================================================
% Dados gerais sobre o trabalho.
%=======================================================================
\autor{Gräbin}{Paulo Henrique Grolli}
\titulo{Um modelo acessível de software para localização e navegação de deficiêntes visuais com Bluetooth Low Energy}
\orientador[Prof.~Dr.]{Costa}{Cristiano André da}
\local{São Leopoldo}
\ano{2015}

%% dados específicos para monografia de Graduação
\unidade{Unidade Acadêmica Graduação}
\curso{Curso de Bacharelado em Ciência da Computação}
\natureza{%
Trabalho de Conclusão de Curso apresentado como requisito parcial
para a obtenção do título de Bacharel em Ciência da Computação
pela Universidade do Vale do Rio dos Sinos --- UNISINOS
}

% cada palavra-chave deve ser fornecida duas vezes, uma em português e
% outra no idioma estrangeiro (na verdade, em tantos idiomas quantos se
% desejar).
\palavrachave{brazilian}{Acessibilidade Ubíqua}
\palavrachave{brazilian}{Deficiência Visual}
\palavrachave{brazilian}{Sistemas de Posicionamento Indoor}
\palavrachave{english}{Ubiquitous Accessibility}
\palavrachave{english}{Visual Impairment}
\palavrachave{english}{Indoor Positioning System}

%=======================================================================
% Início do documento.
%=======================================================================
\begin{document}
%\capa
%\folhaderosto

%=======================================================================
% Dedicatória (opcional).
%=======================================================================
% \begin{dedicatoria}
% À meus pais que sempre me incentivaram na busca por conhecimento \\e a lutar pelos meus objetivos.\\[4ex] % quebra a linha dando um espaçamento maior

% \begin{itshape} % faz o texto ficar em itálico
% Learning is the only thing the mind never exhausts, \\
% never fears, \\
% and never regrets.\\
% \end{itshape}
% --- \textsc{Leonardo Da Vinci} % \textsc é o "small caps"
% \end{dedicatoria}

%=======================================================================
% Agradecimentos (opcional).
%=======================================================================
% \begin{agradecimentos}
% Agradeço primeiramente ao meu orientador Cristiano André da Costa, pessoa a quem eu muito admiro, pelas valiosas conversas, seu auxilio incomparável, e por sua disponibilidade sempre que necessário.

% Aos meus pais, Ivani Grolli Gräbin e Milton Gräbin, que sempre me incentivaram a estudar e a realizar meus sonhos.

% A minha namorada Victoria Caroline da Silva, pela compreensão, por ficar sempre do meu lado e por me motivar a superar os desafios.

% Por fim, aos meus amigos, os quais não diretamente ajudaram na produção desse trabalho, mas estiveram do meu lado nos momentos em que necessitava esvaziar a cabeça e entendiam os convites recusados.

% Muito obrigado!
% \end{agradecimentos}

%=======================================================================
% Epígrafe (opcional).
%
% ``[...] o autor apresenta uma citação, seguida de indicação de autoria,
% relacionada com a matéria tratada no corpo do trabalho. Podem, também,
% constar epígrafes nas folhas de aberturas das seções primárias.''
%=======================================================================
% \begin{epigrafe}
% ``\textit{Ninguém abre um livro sem que aprenda alguma coisa}''.\\
% (Anônimo)
% \end{epigrafe}

%=======================================================================
% Resumo em Português.
%=======================================================================
\begin{abstract}
O presente trabalho apresenta a modelagem de um sistema de posicionamento em ambientes internos, com recursos de acessibilidade, desenhado com o objetivo de ser usado por pessoas com deficiência visual. O modelo consiste em uma aplicação a ser usada em dispositivos móveis, utilizando beacons transmissores Bluetooth espalhados pelo ambiente para obter a localização do usuário. O sistema objetiva fornecer uma localização confiavel e um deslocamento seguro atráves de instruções de voz, leitura de tela e feedback tatíl, para permitir que seus usuários possam se deslocar em ambientes desconhecidos sem a necessidade de auxílio de outras pessoas.
\end{abstract}

%=======================================================================
% Resumo em língua estrangeira
%=======================================================================
\begin{otherlanguage}{english}
\begin{abstract}
This paper presents the modeling of a Indoor Positioning System with accessibility features, designed aiming to be used by visually impaired users. The model consists of an application to be used in mobile devices, making use of Bluetooth beacons deployed in the environment to obtain users location. The system aims to provide a reliable location and a safe and independent navigation using voice guidance, screen reading and haptic feedback in order to allow the users to navigate in unknown environments without need of assistance of other people.
\end{abstract}
\end{otherlanguage}

%=======================================================================
% Lista de Figuras (opcional).
%=======================================================================
%\listoffigures

%=======================================================================
% Lista de Tabelas (opcional).
%=======================================================================
%\listoftables

%=======================================================================
% Lista de Abreviaturas (opcional).
% Deve ser passada como parâmetro a maior das abreviaturas utilizadas.
%=======================================================================
% \begin{listadeabreviaturas}{seg., segs.}
% \item[ampl.] ampliado, -a
% \item[atual.] atualizado, -a
% \item[coord.] coordenador
% \item[N.~T.] Novo Testamento
% \item[seg., segs.] seguinte, -s
% \end{listadeabreviaturas}

%=======================================================================
% Lista de Siglas (opcional).
% Deve ser passada como parâmetro a maior das siglas utilizadas.
%=======================================================================
% \begin{listadesiglas}{IBGE}
% %\item[API] Application Programming Interface
% \item[BLE] Bluetooth Low Energy
% \item[GPS] Global Positioning System
% %\item[HTTP] Hypertext Transfer Protocol
% \item[IBGE] Instituto Brasileiro de Geografia e Estatística
% %\item[JSON] JavaScript Object Notation
% \item[NFC] Near Field Communication
% \item[PDV] Pessoas com Deficiência Visual
% \item[RFID] Radio Frequency Identification
% \item[RLSB]	Royal London Society for Blind People
% %\item[SDK] Software Development Kid
% %\item[SIG] Special Interest Group
% %\item[SOAP] Simple Object Access Protocol
% %\item[TAM] Technology Acceptance Model
% %\item[UML] Unified Modeling Language
% %\item[XML] Extensible Markup Language
% \end{listadesiglas}

%=======================================================================
% Sumário
%=======================================================================
\tableofcontents

%=======================================================================
% Introdução
%=======================================================================
\chapter{Introdução} %Texto contextualizando o problema abordado e o escopo abarcado.
O presente trabalho trata sobre uma alternativa para facilitar as atividades diárias das pessoas com deficiência, especialmente a deficiência visual, atuando na forma com que eles se orientam e deslocam em ambientes internos.
Será mostrada como alternativa a modelagem de um sistema que faz uso de conceitos como dispositivos móveis, acessibilidade ubíqua, tecnologias assistivas e localização por \textit{Bluetooth Low Energy} (BLE).

Como resultado do trabalho espera-se oferecer um sistema de fácil utilização que ofereça orientação e navegação precisa a seus usuários, de maneira que eles tenham confiança e independência. Vale ainda ressaltar que proposito do trabalho não é substituir ferramentas de auxílio como bengala e cães-guia, e sim ser um complemento a esses recursos.

	\section{Motivação}
	%Texto descrevendo a motivação para o trabalho, destacando a oportunidade de pesquisa encontrada e a motivação científica. Nessa seção cabe uma figura mostrando o problema atual e como será resolvido pelo trabalho sendo realizado.

	% Na introdução você deve contextualizar o problema e mostrar por que vale a pena resolvê-lo. Você deve apresentar a solução proposta e mostrar o seu diferencial em relação aos trabalhos relacionados. Observe, porém, que na introdução você deve apenas tratar do O QUÊ e PORQUÊ, sem tratar do como \cite{Hexsel11}, que deve ser explicado na seção que descreve o trabalho desenvolvido.

De acordo com o levantado por \citetexto{IBGE2010} no último Censo Demográfico, realizado em 2010, o Brasil possuía 45,6 milhões de pessoas que afirmaram ser portadoras de deficiência. Desse total, 35,7 milhões, ou 78,2\%, são pessoas com deficiência visual (PDV). A mesma pesquisa leva em consideração as diferentes intensidades da deficiência visual, aplicando os seguintes critérios \cite{IBGE2010}: 

\begin{quote}
	Foi pesquisado se a pessoa tinha dificuldade permanente de enxergar (avaliada com o uso de óculos ou lentes de contato, no caso da pessoa utilizá-los), aplicando os seguintes critérios:
		
	• Não consegue de modo algum - para a pessoa que declarou ser permanentemente incapaz de enxergar; \\
	• Grande dificuldade - para a pessoa que declarou ter grande dificuldade permanente de enxergar, ainda que usando óculos ou lentes de contato; \\
	• Alguma dificuldade - para a pessoa que declarou ter alguma dificuldade permanente de enxergar, ainda que usando óculos ou lentes de contato; ou \\
	• Nenhuma dificuldade - para a pessoa que declarou não ter qualquer dificuldade permanente de enxergar, ainda que precisando usar óculos ou lentes de contato.
\end{quote}

Tendo diversas causas possíveis e atingindo pessoas em qualquer idade, a deficiência visual impõe severas dificuldades em todos os aspectos da vida dos que são afetados por ela. Mobilidade, educação, trabalho e vida social são exemplos, apenas para citar alguns. É difícil para uma pessoa com a visão perfeita se colocar no lugar de uma pessoa que não enxerga. \citetexto{quinones2011supporting} afirma que mobilidade e deslocamento são atividades naturais para aqueles que enxergam sem dificuldades, podendo ir e vir com relativa facilidade já que pistas e sinais irão apontar a direção certa caso se percam, se vejam em situações desconfortáveis ou algum obstáculo apareça. Mostrando o outro lado, em \citetexto{URNA2007}, um trabalho que objetiva auxiliar PDV a cruzar ruas e avenidas, são citadas algumas situações usuais que são impraticáveis para PDV, como por exemplo ler as linhas de ônibus que passam por uma determinada parada, ou ainda ler o destino de um ônibus se aproximando. Não poder ver a sinalização na rua ou semáforos em um cruzamento torna perigoso para PDV transitarem desacompanhados, necessitando ter a companhia de amigos ou familiares durante todo o trajeto. Existem estabelecimentos, como por exemplo o metro de Porto Alegre, que colocam uma parte de seus funcionários a disposição dos deficiêntes para guia-los durante seu trajeto. Porem, seu numero é limitado, o que faz com que os usuários precisem esperar até um funcionário estar disponível.

Em 2014 a Royal London Society for Blind People (RLSB), uma organização sem fins lucrativos que busca criar uma diferença que melhore a vida das futuras gerações de PDV, divulgou um manifesto com o resultado de pesquisas sobre perda de visão e tambem para chamar atenção aos desafios enfrentados diariamente que impedem os PDV de usufruir da totalidade dos seus direitos fundamentais. Desafios esses que não são restritos às divisas do Reino Unido, onde a organização se encontra.
De acordo com o documento, perda de visão é uma experiência tão traumática que pode ser comparada à perda de um ente querido, visto que dificilmente eles recebem o apoio emocional necessário. Não conseguir ver causa isolamento social, o que prejudica enormemente o futuro do individuo. Ainda é dito que empresas ou organizações, como lojas, bancos e agentes de viagem, não compreendem as necessidades e habilidades dos PDV, impedindo-os de levar uma vida comum e impossibilitando-os de realizar as atividades que gostariam. O trabalho ainda destaca que frequentemente as pessoas não sabem oferecer assistencia básica como guia-los em espaços publicos como estações de trem. Dentre as áreas citadas como problemáticas é possível destacar a mobilidade urbana, a educação superior e acessibilidade na tecnologia. A publicação trás ainda motivação que levou a sua publicação, informando que 25\% das pessoas com perda de visão estão insatisfeitas com sua vida \cite{YouthManifesto}.

Em relação a mobilidade, \citetexto{YouthManifesto} cita que existe uma variedade de aplicativos para smartphone que auxíliam pessoas a se localizar e deslocar, porem, poucos deles são acessiveis. É afirmando que se essas técnologias fossem projetadas com todas as pessoas em mente mais PDV poderiam transitar com segurança, reduzindo a dependência de outras pessoas. 

Indo ao encontro das iniciativas das RLSB e atuando na área de mobilidade, o modelo proposto visa auxíliar portadores de deficiênvia visual a se localizar e deslocar em ambientes internos, como por exemplo o campus de uma universidade, buscando facilitar o acesso a educação e a vida social disponível nos campi. Usando dispositivos móveis o sistema irá informar ao usuário sua localização atual e recomendar caminhos que o levem ao seu destino, levando em consideração os obstaculos em sua rota.

Tendo a premissa de funcionar a partir do smartphone do usuário, o sistema será projetado para ter usabilidade intuitiva e ser uma parte quase invisível do cotidiano do usuário, aplicando o conceito de computação ubíqua introduzido em \citetexto{Weiser1991}.	

	\section{Questão de pesquisa}
	Como a Computação Ubíqua pode ser aplicada no desenvolvimento de um modelo de sistema que objetiva promover acessibilidade e auxiliar as pessoas com deficiência visual a transitar com confiança e independência em ambientes internos?

	\section{Objetivos}
	Tendo identificado a questão de pesquisa, foram definidos alguns objetivos para esse trabalho, apresentados nas seções subsequentes.
	
		\subsection{Objetivo geral}
		O objetivo principal desde trabalho é especificar, desenvolver e validar um modelo de sistema para localização e orientação de PDV em ambientes internos através de dispositivos móveis, aplicando fundamentos de acessibilidade e conceitos de computação ubíqua, para permitir que seus usuários se localizem e se desloquem em ambientes desconhecidos sem necessidade do auxílio de outras pessoas.

		O sistema não deve exigir uso de hardware dedicado por parte dos usuários, devendo ser leve, discreto e de fácil uso.
		
		\subsection{Objetivos específicos}
		Para atingir o objetivo geral desse trabalho foram definidos alguns os seguintes objetivos especificos:

		\begin{itemize}
			\item Realizar estudo sobre deficiência visual - estudar essa deficiência buscando entender a maneira com que ela altera a vida de seus portadores e como ela afeta suas capacidades, bem como entender as necessidades que surgem a partir dela, para que os PDV possam ter uma vida mais próxima da levada pelas pessoas sem deficiência visual. Conhecer tecnologias assistivas que possam auxiliar a atingir o objetivo geral.

			\item Realizar estudo sobre computação móvel e ubíqua - compreender quais características da computação móvel e ubíqua são desejáveis em um modelo que será usado por PDV, de forma que o modelo de integre com o cotidiano de seus usuários da maneira mais natural e intuitiva possível.

			\item Realizar estudo sobre sistemas de localização e navegação - analisar trabalhos e soluções existentes nessa área buscando entender quais características deles podem ser aproveitas em um sistema para PDV, bem como novas características que devem ser incorporadas para que as necessidades dos PDV sejam atendidas.

			\item Desenvolver um protótipo da solução - modelar e codificar uma parte do modelo proposto, criando um protótipo que utilize o conhecimento adquirido para promover um sistema que auxilie PDV em 
			possibilite a avaliação do modelo.

			\item Avaliar o sistema desenvolvido - testar o sistema com PDV e avaliar, fazendo uso de questionário, a eficácia e eficiência do sistema. O resultado ajudará identificar a aceitação, pontos fortes e possíveis melhorias do sistema, bem como elencar trabalhos futuros relacionados ao tema. (TODO: avaliação com usuários ou com estudo de caso por cenários?) 
		\end{itemize}

	\section{Organização do texto}
O presente trabalho está dividido em seis capítulos. O primeiro introduz esse trabalho de conclusão, bem como apresenta sua contribuição científica e objetivos.
O segundo capítulo apresenta os principais conceitos utilizados para o embasamento desde trabalho, descrevendo computação móvel e ubíqua, tecnologias assistivas, sistemas de localização e dando um panorama geral sobre a deficiência visual.
O terceiro capítulo apresenta os trabalhos relacionados e traça um comparativo entre eles.
No quarto capítulo é feito um detalhamento do modelo proposto neste trabalho, seus principais requisitos e componentes que formam este modelo.
O quinto capítulo descreve a metodologia de pesquisa, materiais utilizados e como será realizado o desenvolvimento do protótipo.
O sexto e ultimo capítulo conclui o trabalho e apresenta as considerações finais, uma comparação com os trabalhos relacionados e sugestões de trabalhos futuros.

%=======================================================================
% Fundamentação teórica
%=======================================================================
\chapter{Referencial Teórico}

	\section{Computação Móvel e Ubíqua}
Tecnologias tão avançadas que deixam de ser um fim em si mesmas e passam a ser um meio para que as pessoas realizem seus afazeres. Tão conectados, tão presentes na nossa rotina e concebidas para naturalmente se integrarem em nossas vidas que deixam de ser percebidas e se colocam no plano de fundo da nossa percepção. É assim que \citetexto{Weiser1991} começa introduzindo o conceito de computação ubíqua. 

Weiser previu um mundo onde computadores deixariam de possuir apenas o tamanho de um notebook que é usado em cima de uma mesa. Um mundo onde computadores seriam pequenos o suficiente para serem embutidos em botões de uma camisa e também grandes o suficiente para ocuparem os ambientes em que convivemos, estudamos ou trabalhamos. Esses computadores conversariam entre si de maneira continua e transparente, de modo que eles e todos os usuários estariam permanentemente conectados entre si, permitindo que serviços estejam acessíveis em todos os lugares e em todos os momentos. Tal definição vai ao encontro do conceito estabelecido por \citetexto{Satyanarayanan2001} para computação móvel como sendo: “Informação na ponta dos dedos em qualquer lugar e em qualquer tempo”.
	
	\section{Tecnologias Assistivas}
Tecnologias assistivas tem papel chave na independência e segurança de pessoas com deficiência. Para os PDV, tecnologias assistivas bem planejadas e bem implementadas podem fazer significante diferença na educação, aceitação social e produtividade. \citetexto{dias2015navpal}.


Diversas ferramentas existem para promover a inclusão de PDV, bem como para aliviar as dificuldades impostas pela deficiência visual. Cães-guia, bengalas e piso tátil são as principais formas usadas pra facilitar locomoção.

\citetexto{Ganz2014} destaca que smartphones são ferramentas extremamente benéficas no dia a dia das PDV, podendo ser utilizados em diversas finalidade, tais como reconhecimento de cédulas de dinheiro, objetos e cores, navegação na internet, leitura de e-mails e comunicação. O trabalho ainda afirma que o uso de smartphones é possível devido à presença de recursos de acessibilidade oferecidos pelos principais sistemas operacionais disponíveis. 

Entre esses recursos, podemos destacar a leitura de telas e os alertas vibratórios, essenciais pra aqueles que não enxergam as telas sensíveis ao toque que equipam a grande maioria dos smartphones disponíveis atualmente. Ao invés de apenas ler o texto sendo exibido, essa funcionalidade também informa sobre os tipos de cada componente, possibilitando ao usuário saber como lidar com cada um.

\citetexto{mau2008blindaid}, após pesquisa realizada em seu trabalho, define o telefone celular como “a peça de tecnologia mais valiosa para os cegos”. 

Segundo o estudo realizado por \citetexto{quinones2011supporting}, PDV possuem o desejo de carregar consigo a menor quantidade possível de equipamento. É necessário criar tecnologias que não sejam um fardo a ser carregado, mas que ofereçam a quantidade apropriada de informações ao usuário. Uma maneira citada pelo autor é a incorporação de tecnologias de navegação em aparelhos que deficientes visuais carreguem consigo normalmente, assim reduzindo o número de objetos que devem ser cuidados.
	
	\section{Sistemas de localização}
Bluetooth é uma tecnologia de comunicação sem fios de curto alcance, lançada comercialmente em 1990, quando teve sua primeira especificação formal divulgada. Desde então diversas modificações foram feitas e a tecnologia passou por diversos aprimoramentos, estando atualmente na versão 4.x, também conhecida como Bluetooth Smart, lançada em 2010. 

O Bluetooth foi criado e atualmente é mantido por um conjunto de empresas conhecido como Bluetooth Special Interest Group (SIG). Inicialmente formado por Ericsson, Intel, Nokia, Toshiba e IBM, o SIG é hoje composto por mais de 20.000 empresas, incluindo Apple, Microsoft, Motorola e Lenovo.

A tecnologia possui duas formas distintas de atuação, apesar de compartilharem entre si alguns pontos em comum. O primeiro e mais antigo modo, disponível desde a primeira especificação, é chamado Basic Rate (BR). O segundo foi introduzido somente na versão 4.0 e é chamado Low Energy (LE) e é o modo que será utilizado pelo modelo que será proposto.

O LE foi criado para permitir produtos que requerem baixíssimo consumo de energia e baixo custo, quando comparados ao outro modo. Assim sendo, esse modo foi pensado para aplicações que exigem pouca troca de informações. Um exemplo de produto viável com a chegada do modo LE são os beacons bluetooth, que consistem em dispositivos do tamanho de uma moeda comum, compostos por um processador, uma bateria e um transmissor, que transmitem informações sobre si em intervalos regulares. Como o sinal dos beacons tem seu alcance limitado a alguns poucos metros, foi introduzido o conceito de micro localização, tornando possível o desenvolvimento de aplicações novas aplicações e serviços onde antes não era possível, tais como lojas e pontos de venda, grandes shows ou eventos, estádios esportivos, localização em ambientes internos, etc.

A imensa maioria dos telefones celulares hoje fazem uso nativo da tecnologia, não exigindo qualquer tipo de equipamento adicional.

\citetexto{taylor2012smart} revelam que diversas tecnologias foram utilizadas anteriormente em modelos de sistema para navegação para PDV. Sonares, Radio Frequency Identification (RFID), Near Field Communication (NFC), bluetooth e Global Positioning System (GPS) são algumas dessas tecnologias. Os autores ainda apontam que apesar de todas elas oferecerem vantagens em suas propostas, todas também possuem pontos fracos:

\begin{itemize}
  \item GPS é normalmente usado para localização em ambientes externos, mas se mostra ineficiente devido à própria natureza das ondas de rádio usadas na tecnologia, quando obstáculos são locados ao redor do usuário. Além disso, GPS não é uma tecnologia ideal para ambientes internos, pois é capaz apenas de marcar um ponto em um mapa, não sendo suficiente para indicar, por exemplo, múltiplos andares em um prédio, cenário muito comum em ambientes indoor. 
  \item Sonares são formas baratas de detecção de objetos e obstáculos, através de frequências acústicas, mas exigem hardware dedicado e não servem para localização.
  \item RFID, juntamente com NFC, exige proximidade de seus emissores para que a comunicação seja estabelecida.
  \item Os autores não chegam a citar nominalmente os pontos fracos da tecnologia bluetooth, mas um grande ponto que pode ser citado é o consumo de bateria causado pelas versões que antecederam a versão 4, também chamada de bluetooth low energy.    \ldots
\end{itemize}

%=======================================================================
% Trabalhos relacionados
%=======================================================================
\chapter{Trabalhos Relacionados}
\epigrafecap{The most dangerous phrase in the language is "We've always done this way"}{Grace Hopper}

Essa atividade consiste em detalhar os trabalhos relacionados. O capítulo começa apresentando como os trabalhos relacionados foram escolhidos. A seguir, existe uma seção descrevendo cada trabalho relacionado. Por fim, uma seção compara os trabalhos relacionados, apresentando a tabela criada na atividade 2 e descrevendo as lacunas que serão abordadas pela presente dissertação.

Neste capítulo são apresentadas 3 ou 4 (TODO) soluções relacionadas com o tema. Estas foram selecionadas por possuírem proposito semelhante bem como características desejadas por esse trabalho. Ao final desde capitulo, no item~\ref{comparacaoTrabs} é apresentado um comparativo entre os trabalhos selecionados, destacando pontos que são relevantes para o desenvolvimento da arquitetura do presente trabalho.

	\section{PERCEPT: Indoor Nagivation for the Blind and Visually Impaired}
Em \citetexto{Ganz2011} e \citetexto{Ganz2012} apresentam um modelo blablabla

		\cite{Ganz2011}

		\citetexto{Ganz2011}
			
		\cite{Ganz2012}

		\citetexto{Ganz2012}

	\section{Tirésias}
	Em \citetexto{Falk2013} é proposto um modelo para acessibilidade ubíqua chamado Tirésias. De acordo com o artigo, o Tirésias é baseado no Hefestos. Esse, por sua vez, procura estabelecer alguns padrões de acessibilidade que possam ser aplicados em diversos tipos de deficiência. Segundo os autores, o Tirésias pode ser compreendido como uma especialização construída em cima do modelo Hefestos para atender as necessidades das pessoas com deficiência visual. 

	Ainda em \citetexto{Falk2013} a arquitetura do Tirésias é descrita em seus componentes, bem como a maneira integrada em que eles operam. Os componentes citados são:

	\begin{itemize} 
		\item agente assistente pessoal
		\item modulo de saída
		\item módulo de entrada
		\item modulo de configurações
	\end{itemize}

	\FloatBarrier
	\begin{figure}[!ht]
		\caption{Arquitetura do modelo Tirésias}
		\label{fig:visaoGeral}
		\centering%
		\begin{minipage}{.6\textwidth}
			\includegraphics[width=\textwidth]{tiresiasArquitetura}
			\fonte{Tirésias: um modelo para acessibilidade ubíqua orientado à deficiência visual}
		\end{minipage}
	\end{figure}
	\FloatBarrier


	O agente é responsável pela comunicação com o Hefestos, através da qual são obtidos perfis de usuários e recursos disponíveis para o suporte a acessibilidade. 
	O modulo de saída é o responsável por passar as informações aos usuários, sendo utilizados duas maneiras de disponibilizar as informações: leitura de tela e alertas vibratórios.



	\section{Trabalho 3}
	
	\section{Comparação entre os trabalhos estudados}\label{comparacaoTrabs}
\begin{table}
	\caption{Tabela comparativa entre os trabalhos relacionados}
	%\label{tab:estacoes}
	\centering%
	\begin{minipage}{.6\textwidth}
		\begin{tabular*}{\textwidth}{ll}
			\hline
			\textbf{Meses} & \textbf{Estações do Ano}\\
			\hline
			21 de março a 21 de junho & Outono\\
			21 de junho a 23 de setembro & Inverno\\
			23 de setembro a 21 de dezembro & Primavera\\
			21 de dezembro a 21 de março & Verão\\
			\hline
		\end{tabular*}
		\fonte{Elaborada pela autora.}
	\end{minipage}
\end{table}


%=======================================================================
% Modelo proposto
%=======================================================================
\chapter{Modelo Proposto}
Essa etapa consiste em detalhar o modelo. O capítulo do modelo deve começar a partir da delimitação da pesquisa, dizendo quais lacunas serão atacadas e como. Geralmente há uma visão geral do modelo aqui, já desenvolvido na atividade 4. Esse capítulo detalha como a questão de pesquisa será respondida? Que modelo computacional é necessário para isso. Se é proposta uma ontologia, ela é detalhada aqui; Se há um modelo de contexto, ele é detalhado aqui. Pode-se usar alguns diagramas para representar a arquitetura, interações, fluxos e componentes. Lembre-se que o modelo pode ter diferentes implementações e não precisa ser limitado por uma tecnologia específica. Por exemplo, não faz sentido dizer que o modelo é para Android ou feito em Java. Por que não poderia ser para iOS e feito em Objective C? Que característica de um modelo limitaria isso?


Esse capítulo apresenta um modelo de sistema que auxilia a navegação de deficientes visuais.




No Brasil, existe legislação especifica para garantir a acessibilidade e os direitos dos PDV. Os Decretos 3.298/99 e 5.296/04 definem critérios técnicos para conceituação de alguém como PDV. Os níveis de deficiência visual são baixa visão e cegueira, baseados na acuidade visual medida através de exame. A Lei 10.098/00 estabelece normas e critérios básicos para a promoção da acessibilidade, descrevendo normas de construção de edifícios públicos, de uso coletivo e privados, bem como regulamentando como deve se dar a acessibilidade nos sistemas de comunicação e sinalização. Porem, a vida digital não é contemplada pela legislação brasileira.


\citetexto{quinones2011supporting} busca entender, através de entrevistas, como PDV realizam seus deslocamentos diários, bem como elencar requisitos necessários para desenvolver sistemas que os auxiliam nesta tarefa. O trabalho afirma que PDV criam mapas mentais ricos em detalhes dos locais que mais frequentam, memorizando rotas e marcos específicos destes locais. É sugerido no trabalho que uma característica importante de um sistema para navegação é o suporte a ambientes dinâmicos, no sentido de que mesmo caminhos conhecidos podem sofrer alterações, mesmo que levemente.




Vini:
Modelo Proposto
	Visão Geral do Modelo
	Arquitetura Proposta
	Requisitos e Casos de Uso
		Requisitos Funcionais
		Requisitos Não Funcionais
	Modelo do Prontuario Proposto


Plets:
Modelo Proposto
	Requisitos
		Requisitos Funcionais
		Requisitos Não Funcionais
	Arquitetura
	Modelo de Dados
	Modelo de Segurança
		Senha
		Identificação de uso indevido
		Criptografia
	Modelo de comunicação
	Modelo de contexto


Visando mitigar o impacto da deficiência visual na vida dos usuários, o modelo proposto fará uso de dispositivos móveis aliados aos conceitos da computação ubíqua para auxiliar os PDV durante o deslocamento em ambientes internos, tais como o campus da UNISINOS, através de instruções dadas por voz e alertas vibratórios. O modelo oferecerá as funcionalidades de informar a localização atual, pesquisar locais, mostrar locais de conveniência (restaurantes, agências bancárias, etc), informar rota mais curta até o local selecionado e salvar local como favorito, entre outras. Para as funcionalidades espaciais, o modelo usará beacons transmissores bluetooth previamente espalhados pelo campus.

O modelo objetiva oferecer localização confiável a seus usuários através de seus smartphones sem necessidade de hardware dedicado, assim possibilitando navegação segura e independente, tendo baixo custo e facilidade de implantação.

	\section{Visão Geral}

A Figura~\ref{fig:blablabla} mostra ilustra as fases psicológicas da escrita da dissertação. Você vai se reconhecer no personagem. ;-)

\FloatBarrier
\begin{figure}[!ht]
	\caption{Visão geral do modelo}
	\label{fig:blablabla}
	\centering%
	\begin{minipage}{.8\textwidth}
		\includegraphics[width=\textwidth]{escrita}
		\fonte{Elaborado pelo autor.}
	\end{minipage}
\end{figure}
\FloatBarrier

	\section{Requisitos}

	\section{Arquitetura}
 

\FloatBarrier 
\begin{figure}[!ht]
	\caption{Diagrama de blocos da arquitetura}
	\label{fig:arquitetura}
	\centering%
	\begin{minipage}{.8\textwidth}
		\includegraphics[width=\textwidth]{arquitetura.png}
		\fonte{Elaborado pelo autor.}
	\end{minipage}
\end{figure}
\FloatBarrier 




%=======================================================================
% Metodologia
%=======================================================================
\chapter{Metodologia}
Enquadrando o presente trabalho na classificação apresentada em \citetexto{Gerhardt2009}, quanto à natureza da pesquisa, ele se caracteriza como uma pesquisa aplicada, pois objetiva gerar conhecimentos para aplicação prática dirigidos a solução de um problema específico que é a navegação e localização de deficientes visuais. A abordagem do trabalho é qualitativa e, do ponto de vista dos objetivos, é uma pesquisa exploratória e descritiva, buscando resultar em um modelo acessível de localização e navegação em ambientes internos. O procedimento técnico utilizado para a construção da pesquisa é a pesquisa bibliográfica.

	\section{Desenvolvimento}
O trabalho de modelagem de desenvolvimento do xxxxxxx (TODO) foi divido em (TODO) XX passos metodológicos apresentados na figura (TODO) XX. Os passos 1 a X foram realizados no presente trabalho. Os demais passos serão realizados na disciplina Trabalho de Conclusão II, que também compreende os requisitos para a colação de grau da Universidade do Vale do RIo dos Sinos (UNISINOS).

A seguir são apresentados brevemente cada um dos passos concebidos:

% passos
\begin{enumerate}
	%	1
	\item \textbf{Revisar bibliografia:} teste teste teste
	%	2
    \item \textbf{Realizar estudo sobre deficiência visual:} teste teste teste
	%	3
    \item \textbf{Realizar estudo sobre sistemas de localização e navegação:} teste teste teste
	%	4
    \item \textbf{Realizar estudo sobre tecnologias assistivas:} teste teste teste
	%	5
    \item \textbf{Realizar estudo sobre computação móvel e ubíqua:} teste teste teste
    %	6
    \item \textbf{Elencar requisitos e modelar o sistema:} teste teste teste
    %	7
    \item \textbf{Codificar sistema:} teste teste teste
	%	8
	\item \textbf{Testar aceitação do sistema:} teste teste teste	
	%	9
	\item \textbf{Analisar resultados dos testes:} teste teste teste
\end{enumerate}

\begin{description}
	\item [1. Revisar bibliografia] teste teste teste
	%	2
    \item [2. Realizar estudo sobre deficiência visual] teste teste teste
	%	3
    \item [3. Realizar estudo sobre sistemas de localização e navegação] teste teste teste
	%	4
    \item [Realizar estudo sobre tecnologias assistivas] teste teste teste
	%	5
    \item [Realizar estudo sobre computação móvel e ubíqua] teste teste teste
    %	6
    \item [Elencar requisitos e modelar o sistema] teste teste teste
    %	7
    \item [Codificar sistema] teste teste teste
	%	8
	\item [Testar aceitação do sistema] teste teste teste	
	%	9
	\item [Analisar resultados dos testes] teste teste teste
\end{description}

	\section{Avaliação}
A avaliação do modelo proposto se dará após a implementação e implantação do sistema. Para a implantação do mesmo serão convocados usuários PDV com interesse em colaborar com a pesquisa. Os usuários utilizarão o sistema e posteriormente o avaliarão usando duas técnicas distintas. A primeira será através de questionários anônimos durante o período de testes, afim de detectar problemas pontuais e verificar a adaptação do usuário ao sistema durante o período de uso. A segunda será a técnica da entrevista estruturada. 

A entrevista ocorrerá em dois momentos. Antes de iniciar o período de testes para obter informações sobre a forma como os usuários se guiam quando necessitam deslocar-se em um ambiente desconhecido. Após o termino dos testes para obter informações relativas a qualidade da localização e da navegação oferecida pelo protótipo. O resultado da analise permitirá identificar a eficácia do sistema e possibilitará elencar melhorias e/ou trabalhos relacionados ao tema.









\chapter{Conclusão}

	\section{Comparação entre os trabalhos estudados e o modelo proposto}

\begin{table}
	\caption{Comparação entre os trabalhos estudados e o modelo proposto}
	%\label{tab:estacoes}
	\centering%
	\begin{minipage}{.6\textwidth}
		\begin{tabular*}{\textwidth}{ll}
			\hline
			\textbf{Meses} & \textbf{Estações do Ano}\\
			\hline
			21 de março a 21 de junho & Outono\\
			21 de junho a 23 de setembro & Inverno\\
			23 de setembro a 21 de dezembro & Primavera\\
			21 de dezembro a 21 de março & Verão\\
			\hline
		\end{tabular*}
		\fonte{Elaborada pela autora.}
	\end{minipage}
\end{table}
	
	\section{Trabalhos futuros}

Em caso de dúvida, siga as orientações do manual da Biblioteca \cite{Biblioteca11} e, se necessário, da norma NBR~6023 \cite{NBR6023:2002}.


























































Geralmente, a introdução tem uma estrutura similar ao resumo e deve apresentar:
\begin{itemize}
	\item \textbf{Contexto e motivação:} Aqui você deve apresentar o contexto do trabalho (área de que ele se trata) e uma motivação para trabalhar nesse assunto.
	\item \textbf{Problema:} Aqui você vai apresentar um problema, uma lacuna, observada na área e que você pretende tratar. Você deve se perguntar aqui: ``Que respostas estou disposto a responder?''. O problema deve ser definido claramente e delimitado em termos de espaço de tempo. Veja que essa parte visa alertar o leitor de que o que você está propondo é uma solução para um problema observado na área. 
	

	\item \textbf{Objetivos:} Aqui você deve apresentar os objetivos do seu trabalho. Tome cuidado para não confundir objetivos com atividades.   Faça a si mesmo a pergunta: ``O que pretendo alcançar com a pesquisa?''. Você pode discernir entre objetivos gerais e objetivos específicos:
	\begin{itemize}
		\item Objetivo geral --- qual o propósito da pesquisa?
		\item Objetivos específicos --- abertura do objetivo geral em outros menores (possíveis capítulos).
	\end{itemize}
	
\end{itemize}

%=======================================================================
% Referências
%=======================================================================
\bibliography{bibliografia}

\end{document}